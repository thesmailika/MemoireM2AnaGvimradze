% Chapter Template

\chapter{Autres modèles et architectures} % Main chapter title

\label{Chapter5} % Change X to a consecutive number; for referencing this chapter elsewhere, use \ref{ChapterX}

%----------------------------------------------------------------------------------------
%	SECTION 1
%----------------------------------------------------------------------------------------

\section{CogAff}

CogAff est un modèle de traitement de l'information proposé par Sloman \parencite{sloman2005architectural}. Il comprend trois niveaux, à savoir: réactif, délibératif et réflexif (méta-gestion). 
Chaque niveau comprend des mécanismes de perception, des mécanismes de traitement centralisés et des mécanismes d'action. L'architecture possède un mécanisme d'alarme ainsi que des communications d'informations entre toutes les parties de l'architecture. Le mécanisme d'alarme fonctionne de manière centralisée et s'apparente au mécanisme d'interruption.





%-----------------------------------
%	SUBSECTION 1
%-----------------------------------
\subsection{H-CogAff}

H-CogAff est un exemple particulier de CogAff expliquant les phénomènes mentaux humains. Chaque niveau supporte différentes catégories d'émotions. «D'autres subdivisions sont nécessaires pour couvrir toute la variété des émotions humaines, d'autant plus que les émotions peuvent changer de caractère au fil du temps, à mesure qu'elles grandissent» \parencite{sloman2005architectural}.


\section{CLARION}

CLARION a une structure similaire aux trois niveaux de traitement de l’information. C'est une architecture cognitive qui comporte deux niveaux: le niveau implicite (similaire aux niveaux réactifs et de routines) et le niveau explicite (similaire au niveau réflexif). Le niveau implicite est le niveau inférieur et code les connaissances implicites. Il utilise un réseau de neurones multi-couches avec Q-learning pour acquérir des connaissances implicites. Le niveau explicite est le niveau supérieur et code la connaissance explicite. Il utilise un apprentissage ponctuel pour acquérir des connaissances explicites.

~\par
Le niveau le plus bas et le niveau le plus élevé échangent leurs apprentissages au moyen d'un apprentissage ascendant et descendant. Chaque niveau comprend quatre sous-systèmes fonctionnels distincts:

\begin{enumerate}
\item un sous-système centré sur l'action pour contrôler les actions;
\item un sous-système non centré sur l'action pour conserver les connaissances générales implicites ou explicites;
\item un sous-système méta-cognitif pour surveiller, diriger et modifier les opérations de tous les sous-systèmes;
\item un sous-système de motivation pour fournir les motivations sous-jacentes de la perception, de l'action et de la cognition, en termes d'impulsion et de rétroaction.
\end{enumerate}

\section{SHAME}

SHAME (Architecture évolutive et hybride pour le mimétisme des émotions) est un modèle émotionnel ou système simulant l'état émotionnel d'un agent agissant dans un environnement virtuel présenté par Kesteren en 2001 \parencite{kesteren2001supervised}. Il a construit un modèle de simulation à base d'agents appelés «GridWorld». SHAME implémente la fonction d’affect similaire à la fonction d’affect dans le niveau de réflexion et des fonctions de motivation et de comportement similaires à celles du niveau de réactif. Il utilise un réseau de neurones pour apprendre comment l'état émotionnel devrait être influencé par la survenue de stimuli.


\section{Zamin}

Zamin est un environnement de simulation de vie artificielle pour évaluer les capacités des agents à produire un comportement émotionnel et à prendre de meilleures décisions, développé par Zadeh, Shouraki et Halavati \parencite{zadeh2006emotional}. Ces derniers ont mis en place cet environnement afin d’étudier le possible rôle des émotions dans la gestion des ressources mentales. Ils utilisent uniquement un comportement positif/négatif et un comportement d'approche/d'évitement (uniquement la fonctionnalité du niveau réactif) dans un environnement de type prédateur-proie.
