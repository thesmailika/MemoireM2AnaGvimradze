\chapter{Conclusion} % Main chapter title

\label{Chapter4} % Change X to a consecutive number; for referencing this chapter elsewhere, use \ref{ChapterX}

Pour conclure, nous pouvons constater à partir des recherches faites sur les SMA (systèmes multi-agents) ainsi que leurs domaines d’applications, et les différents modèles de programmation avec lesquels celles-ci sont conçues, telle que le modèle BDI (section \ref{bdiA}) et l'exemple du “SWARMM” (\ref{ssec:swarm}) que la plupart des agents autonomes  actuellement en opération procèdent de manière très rationnelle dans leur prise de décisions.

~\par
Cela se fait suite à une sélection de multiples possibilités, et élection de celle ayant le plus haut “score” qui est aussi celle ayant prouvé son fonctionnement dans les situations précédentes équivalentes à celle en cours d'exécution.

~\par
Ainsi, différents chercheurs se sont penchés sur la question afin de rendre ces agents plus réalistes et crédible aux yeux de l’homme. Leurs recherches, d’abord sur l’homme ont prouvé que ce dernier n'effectue un choix rationnel dans une situation donnée, au sein d’un environnement donné que très rarement. Ils ont en conclue différents modèles dit “Naturels” ou NDM “prise de décisions en milieu naturel” basés sur le comportement des agents humains, notamment ceux qui sont les plus compétents dans le domaine sur lequel le modèle est appliqué (médical, militaire, aéronautique, etc.).

~\par
Le but était ensuite de pouvoir appliquer ces mêmes schémas de pensées et de sélections à des agents autonomes, de nombreux modèles ont vu le jour suite à ces recherches, principalement destinés à l’aide à la décision, ou pour développer de meilleurs modèles cognitifs pour ces agents lors des simulations.

~\par
Ma contribution dans ce domaine a montré qu’il était possible d’établir certaines règles simples dans un environnement donné constitué d’agents autonomes, qui permettent de simuler des émotions, basées sur des facteurs physiques, de distance, temporels ou encore de prédation. Ces règles peuvent influer sur le comportement de ces agents de manière concrète, c’est-à-dire engendrer des actions suite à des changements dans leur environnement, changements qui influent sur leur état ou émotion du moment et les poussent à repenser leur monde où en tout cas, la vision qu’ils s’en font, et donc de penser un nouveau plan d’action selon l’évolution de leurs émotions.

~\par
L’exemple que j’ai pris peut-être très facilement amélioré en intégrant des données plus détaillées sur les émotions, comme des facteurs de temps, qui peuvent être issus de la psychologie humaine lors de recherches sur les émotions sur ce dernier, ou encore des facteurs d'intensités, qui peuvent être déduits de situations réelles. ce qui rendrait le passage d’une émotion à une autre plus réaliste et donc plus crédible à l’œil de l’homme.

~\par
Selon les retours de Souda et d'Hajar que j'ai pu avoir, je peux conclure que ma contribution pour le jeu vidéo “Do not shoot Aliens” a eu un résultat. Même si c'est loin d'être parfait, nous avons déjà pu obtenir des idées d'utilisateurs pour les axes d'amélioration.

~\par
Le plus important, à mon avis, est qu'on a pu voir très clairement qu'en ajoutant des critères plus en plus précises, il y a la possibilité de simuler le comportement des êtres humains avec les IA.